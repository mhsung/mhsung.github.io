%%%%%%%%%%%%%%%%%%%%%%%%%%%%%%%%%%%%%%%%%
% Plasmati Graduate CV
% LaTeX Template
% Version 1.0 (24/3/13)
%
% This template has been downloaded from:
% http://www.LaTeXTemplates.com
%
% Original author:
% Alessandro Plasmati (alessandro.plasmati@gmail.com)
%
% License:
% CC BY-NC-SA 3.0 (http://creativecommons.org/licenses/by-nc-sa/3.0/)
%
% Important note:
% This template needs to be compiled with XeLaTeX.
% The main document font is called Fontin and can be downloaded for free
% from here: http://www.exljbris.com/fontin.html
%
%%%%%%%%%%%%%%%%%%%%%%%%%%%%%%%%%%%%%%%%%

%----------------------------------------------------------------------------------------
%	PACKAGES AND OTHER DOCUMENT CONFIGURATIONS
%----------------------------------------------------------------------------------------

%\documentclass[a4paper,10pt]{article} % Default font size and paper size
\documentclass[letterpaper,10pt]{article} % Default font size and paper size

\usepackage{fontspec} % For loading fonts
%\defaultfontfeatures{Mapping=tex-text}
%\setmainfont[SmallCapsFont = Fontin SmallCaps]{Fontin} % Main document font
%\font\fb=''[cmr10]'' % Change the font of the \LaTeX command under the skills section
% FONTS
\defaultfontfeatures{Mapping=tex-text} % converts LaTeX specials (``quotes'' --- dashes etc.) to unicode
\setromanfont[Scale=1.2]{Crimson Text}
%\setmonofont[Scale=0.8]{Monaco} 
%\setsansfont[Scale=0.9]{Optima Regular} 
% ---- MARGIN YEARS
\newcommand{\years}[1]{\marginpar{\scriptsize #1}}

\usepackage{xunicode,xltxtra,url,parskip} % Formatting packages

\usepackage[usenames,dvipsnames]{xcolor} % Required for specifying custom colors

%\usepackage[big]{layaureo} % Margin formatting of the A4 page, an alternative to layaureo can be
%\usepackage{fullpage}
\usepackage[margin=0.75in]{geometry}
% To reduce the height of the top margin uncomment: \addtolength{\voffset}{-1.3cm}

\usepackage{hyperref} % Required for adding links	and customizing them
%\definecolor{linkcolor}{rgb}{0,0.2,0.6} % Link color
%\definecolor{linkcolor}{rgb}{0.5075,0,0} % Link color
\definecolor{linkcolor}{rgb}{0,0.2549,0.5686} % Link color
\hypersetup{colorlinks,breaklinks,urlcolor=linkcolor,linkcolor=linkcolor} % Set link colors throughout the document

\usepackage{titlesec} % Used to customize the \section command
\titleformat{\section}{\large\scshape\raggedright}{}{0em}{}[\titlerule] % Text formatting of sections
\titlespacing{\section}{0pt}{3pt}{3pt} % Spacing around sections

% To add some paragraph space between lines.
% This also tells LaTeX to preferably break a page on one of these gaps
% if there is a needed pagebreak nearby.
\newcommand{\blankline}{\quad\pagebreak[2]}

\usepackage{makecell}

\begin{document}

%\pagestyle{empty} % Removes page numbering

%----------------------------------------------------------------------------------------
%	CONTACT INFORMATION
%----------------------------------------------------------------------------------------

\par{\centering{\huge Minhyuk Sung}\bigskip\par} % Your name

\par{\centering{\large
Assistant Professor,
\href{http://cs.kaist.ac.kr/}{School of Computing},
\href{http://www.kaist.ac.kr/}{KAIST}
}\bigskip\par}

\par{\centering{
\begin{tabular}{c|rl}
  N1, Room 607                      & \textsc{Phone:} & +82-42-350-3587 \\
  291 Daehak-ro, Yuseong-gu         & \textsc{Email:} & \href{mailto:mhsung@kaist.ac.kr}{mhsung@kaist.ac.kr} \\
  Daejeon, 34141, Republic of Korea & \textsc{Website:} & \href{https://mhsung.github.io}{https://mhsung.github.io}
\end{tabular}
}\bigskip\par}

%\section{Contact Information}
%\begin{tabular}{l|r}
%\textsc{Address:} & Clark Center, S297, 318 Campus Drive, Stanford, CA 94305 \\
%%\textsc{Phone:} & 650-391-6340\\
%\textsc{Email:} & \href{mailto:mhsung@cs.stanford.edu}{mhsung@cs.stanford.edu} \\
%%\textsc{Website:} & \href{https://mhsung.github.io/}{https://mhsung.github.io/}
%\textsc{Website:} & \href{https://cs.stanford.edu/~mhsung}{https://cs.stanford.edu/\symbol{126}mhsung}
%\end{tabular}

\blankline


%----------------------------------------------------------------------------------------
%	RESEARCH INTERESTS
%----------------------------------------------------------------------------------------

\section{Research Interests}
3D Machine Learning, Geometry Processing, Computer Graphics, Computer Vision.\\
\blankline


%----------------------------------------------------------------------------------------
%	EDUCATION
%----------------------------------------------------------------------------------------

\section{Education}

\begin{tabular}{r|p{14cm}}
2013 - 2019 & Ph.D. in \textsc{Computer Science},\\
& \href{https://www.stanford.edu}{\textbf{Stanford University}} \\
& Stanford, CA, USA\\
& Dissertation: \href{https://searchworks.stanford.edu/view/13333384}{\textbf{Learning and exploring the compositional structure of 3D data}}\\
& Advisor: \href{https://geometry.stanford.edu/member/guibas/}{Leonidas Guibas}\\
&\\

%------------------------------------------------

2008 - 2010 & Master of Science in \textsc{Computer Science},\\
& \href{https://www.kaist.ac.kr/en/}{\textbf{Korea Advanced Institute of Science and Technology (KAIST)}} \\
& Daejeon, South Korea\\
& Thesis: \href{http://koasas.kaist.ac.kr/handle/10203/34894}{\textbf{A Spectral Approach to Shape Matching Using a Heat Kernel Function}}\\
& Advisor: Sung Yong Shin\\
&\\

%------------------------------------------------

2004 - 2008 & Bachelor of Science in \textsc{Computer Science},\\
& \href{https://www.kaist.ac.kr/en/}{\textbf{Korea Advanced Institute of Science and Technology (KAIST)}} \\
& Daejeon, South Korea\\
& \emph{Top Rank} in Computer Science Department\\
%& Summa Cum Laude | \textsc{Gpa}: 4.07/4.3\\
%&\\

%------------------------------------------------

%2004& \textbf{Hansung Science High School}, Seoul, South Korea\\
\end{tabular}\\

\blankline


%----------------------------------------------------------------------------------------
%	EMPLOYMENT
%----------------------------------------------------------------------------------------

\section{Employment}

Assistant Professor \hfill{\textsc{Jan 2021 - Present}\\
\href{http://cs.kaist.ac.kr/}{\textbf{School of Computing}}, \href{http://www.kaist.ac.kr/}{\textbf{KAIST}}, Daejeon, Republic of Korea

%------------------------------------------------

Research Scientist \hfill{\textsc{Oct 2019 - Dec 2020}}\\
\href{https://research.adobe.com}{\textbf{Adobe Research}}, San Jose, CA, USA 

%------------------------------------------------

Research Intern \hfill{\textsc{Jun 2017 - Sep 2017}}\\
\href{https://www.autodeskresearch.com}{\textbf{Autodesk Research}}, San Francisco, CA, USA
%	\begin{itemize}
%		\item Project: Learning and Exploiting Component and Assembly Variability in Design
%	\end{itemize}

%------------------------------------------------

Research Intern \hfill{\textsc{Jun 2016 - Sep 2016}}\\
\href{https://research.adobe.com/}{\textbf{Adobe Research}}, Seattle, WA, USA
%	\begin{itemize}
%		\item Project: 3D Object Priors for Visual Odometry
%	\end{itemize}

%------------------------------------------------

Research Intern \hfill{\textsc{Jun 2015 - Sep 2015}}\\
\href{http://www.google.com}{\textbf{Google}}, Mountain View, CA, USA
%	\begin{itemize}
%		\item Developed a global-sampling-based remeshing algorithm.
%	\end{itemize}

%------------------------------------------------

Research Intern \hfill{\textsc{Jun 2014 - Sep 2014}}\\
\href{http://www.google.com}{\textbf{Google}}, Mountain View, CA, USA
%	\begin{itemize}
%    %\item Developed mesh simplification and mesh texturing algorithms for color and depth images scanned by `Project Tango' mobile devices.
%		\item Developed mesh simplification and texturing algorithms in `Project Tango' team.
%	\end{itemize}

%------------------------------------------------

Researcher \hfill{\textsc{Mar 2010 - Jul 2013}}\\
\href{https://www.imrc.kist.re.kr/en/}{Imaging Media Research Center (IMRC)}\\
\href{https://eng.kist.re.kr/}{\textbf{Korea Institute of Science and Technology (KIST)}}, Seoul, South Korea\\
%	\begin{itemize}
%		\item Developed a novel framework of 3D scene reconstruction using a single color image and a sparse depth point set
%		[\ref{icip13}].
%		
%		\item Developed a novel shape correspondence algorithm discovering multiple plausible matching configurations under the
%		symmetry ambiguity [\ref{cg13}, \ref{ms10}].
%		
%		\item Developed a 3D color object reconstruction system using Kinect [\ref{isuvr12}].\\
%		(Contributed to generate a smooth mesh surface based on acquired 3D voxel data, project multi-view color images on the
%		reconstructed 3D mesh.)\\
%		YouTube: \href{http://youtu.be/qNlm01k7_hQ}{IMRC Gangnam Style (Kinect 3D Modeling \& Character Animation)}
%		
%		\item Developed an augment reality system detecting multiple 2D planar objects real-time on mobile platforms (iOS,
%		Android, Unity 3D).\\
%		Launched an augmented reality tour guide service in National Palace Museum of Korea.\\
%		Press (in Korean, Translated): \href{http://goo.gl/1bFGiQ}{Hankook Economy Newspaper}, Jul 31. 2012
%		
%		\item Implemented PTAM (Parallel Tracking and Mapping) on Android platform [\ref{kjmr11}].\\
%		(Co-worked with Visual Computing Lab., Kookmin University (Prof. Junho Kim))\\
%		YouTube: \href{http://youtu.be/drSEK8k3QJ0}{PTAM on Android Phone (Team AVatAR)}
%	\end{itemize}
\blankline

%%------------------------------------------------
%
%Research Assistant \hfill{\textsc{Sep 2008-Jan 2010}}\\
%Computer Grahpics Lab., \textbf{KAIST}, Seoul, Korea
%	\begin{itemize}
%		\item  Developed a triangular-invariant algorithm extending the previous OUM(Ordered
%Upwind Method)-based anisotropic geodesic solver on 2-manifolds [\ref{tvcg12}].
%	\end{itemize}
%\blankline


%%----------------------------------------------------------------------------------------
%%	WORK EXPERIENCE
%%----------------------------------------------------------------------------------------
%
%\section{Work Experience}
%
%\begin{tabular}{r|p{11cm}}
%%\textsc{Jun 2008-Aug 2008} & Software Engineer Intern\\
%%& \textbf{Association for International Practical Training (AIPT)},\\
%%& Columbia, MD, USA\\
%%& (Currently changed to Cultural Vistas)\\
%%& \small Designed database of internship programs and applicants information.\\
%%& \\
%
%%------------------------------------------------
%
%\textsc{Dec 2007-Apr 2008} & Intern\\
%& \textbf{Sony Computer Entertainment Korea (SCEK)}, Seoul, South Korea\\
%& \small Developed a visual GUI editor on PlayStation\textregistered 3 platform.\\
%& \\
%
%%------------------------------------------------
%
%\textsc{Apr 2007-Jul 2007} & Undergraduate Research Assistant\\
%& Virtual Media Lab., \textbf{KAIST}, Daejeon, South Korea\\
%& \small Developed a 3ds Max\textregistered 9.0 plug-in of automatic facial rigging system.\\
%& \\
%
%%------------------------------------------------
%
%\textsc{Jul 2006-Aug 2006} & Undergraduate Research Assistant\\
%& \textbf{Information and Communications University (ICU)}, Daejeon, South Korea\\
%& (Currently changed to KAIST-ICC)\\
%& \small Designed a teleconference software GUI based on AccessGrid\textregistered framework and Tcl/Tk.\\
%& \\
%
%%------------------------------------------------
%
%\textsc{Dec 2007-Apr 2008} & Undergraduate Research Assistant\\
%& \textbf{Fudan University}, Shanghai, China\\
%& \small Participated in research of face recognition.\\
%\end{tabular}
%
%\blankline

\newpage


%----------------------------------------------------------------------------------------
%	PUBLICATIONS
%----------------------------------------------------------------------------------------

\section{Publications}

\begin{enumerate}

\item \label{iccv21_2}
\href{https://arxiv.org/abs/2109.00113}{\textbf{CPFN: Cascaded Primitive Fitting Networks for High-Resolution Point Clouds}}\\
Eric-Tuan Lê, \textbf{Minhyuk Sung}, Duygu Ceylan, Radomír Měch, Tamy Boubekeur, Niloy Mitra\\
ICCV 2021\\
\blankline

\item \label{iccv21_1}
\href{https://arxiv.org/abs/2109.02259}{\textbf{CTRL-C: Camera calibration TRansformer with Line-Classification}}\\
Jinwoo Lee, Hyunsung Go, Hyunjoon Lee, Sunghyun Cho, \textbf{Minhyuk Sung}, Junho Kim\\
ICCV 2021\\
\blankline

\item \label{cvpr21_3}
\href{https://mhsung.github.io/papers/deep-meta-handles.html}{\textbf{DeepMetaHandles: Learning Deformation Meta-Handles of 3D Meshes with Biharmonic Coordinates}}\\
Minghua Liu, \textbf{Minhyuk Sung}, Radomír Měch, Hao Su\\
CVPR 2021 (Oral)\\
\blankline

\item \label{cvpr21_2}
\href{https://cg.cs.tsinghua.edu.cn/people/~huangjh/publication/multibodysync/}{\textbf{MultiBodySync: Multi-Body Segmentation and Motion Estimation via 3D Scan Synchronization}}\\
Jiahui Huang, He Wang, Tolga Birdal, \textbf{Minhyuk Sung}, Federica Arrigoni, Shi-Min Hu, Leonidas Guibas\\
CVPR 2021 (Oral)\\
\blankline

\item \label{cvpr21_1}
\href{https://joint-retrieval-deformation.github.io/}{\textbf{Joint Learning of 3D Shape Retrieval and Deformation}}\\
Mikaela Angelina Uy, Vladimir G. Kim, \textbf{Minhyuk Sung}, Noam Aigerman, Siddhartha Chaudhuri, Leonidas Guibas\\
CVPR 2021\\
\blankline

\item \label{siggraphasia20}
\href{https://mhsung.github.io/papers/deform-sync-net.html}{\textbf{DeformSyncNet: Deformation Transfer via Synchronized Shape Deformation Spaces}}\\
\textbf{Minhyuk Sung}*, Zhenyu Jiang*, Panos Achlioptas, Niloy J. Mitra, Leonidas J. Guibas\\
(* Equal contribution)\\
SIGGRAPH Asia 2020\\
\blankline

\item \label{eccv20_4}
\href{https://deformscan2cad.github.io/}{\textbf{Deformation-Aware 3D Shape Embedding and Retrieval}}\\
Mikaela Angelina Uy, Jingwei Huang, \textbf{Minhyuk Sung}, Tolga Birdal, Leonidas Guibas\\
ECCV 2020\\
\blankline

\item \label{eccv20_3}
\href{https://arxiv.org/abs/2007.11855}{\textbf{Neural Geometric Parser for Single Image Camera Calibration}}\\
Jinwoo Lee, \textbf{Minhyuk Sung}, Hyunjoon Lee, Junho Kim\\
ECCV 2020\\
\blankline

\item \label{eccv20_2}
\href{https://geometry.stanford.edu/projects/pix2surf/}{\textbf{Pix2Surf: Learning Parametric 3D Surface Models of Objects from Images}}\\
Jiahui Lei, Srinath Sridhar, Paul Guerrero, \textbf{Minhyuk Sung}, Niloy Mitra, Leonidas Guibas\\
ECCV 2020\\
\blankline

\item \label{eccv20_1}
\href{https://cs.stanford.edu/~kaichun/impartass/}{\textbf{Learning 3D Part Assembly from a Single Image}}\\
Yichen Li*, Kaichun Mo*, Lin Shao, \textbf{Minhyuk Sung}, Leonidas Guibas\\
(* Equal contribution)\\
ECCV 2020\\
\blankline

\item \label{cvpr19_2}
\href{https://arxiv.org/abs/1811.08988}{\textbf{Supervised Fitting of Geometric Primitives to 3D Point Clouds}}\\
Lingxiao Li*, \textbf{Minhyuk Sung}*, Anastasia Dubrovina, Li Yi, Leonidas Guibas\\
(* Equal contribution)\\
CVPR 2019 (Oral)\\
%\href{https://arxiv.org/abs/1811.08988}{arxiv:1811.08988}\\
\blankline

\item \label{cvpr19_1}
\href{https://arxiv.org/abs/1812.03320}{\textbf{GSPN: Generative Shape Proposal Network for 3D Instance Segmentation in Point Cloud}}\\
Li Yi, Wang Zhao, He Wang, \textbf{Minhyuk Sung}, Leonidas Guibas\\
CVPR 2019\\
%\href{https://arxiv.org/abs/1812.03320}{arXiv:1812.03320}\\
\blankline

\item \label{neurips18}
\href{https://arxiv.org/abs/1805.09957}{\textbf{Deep Functional Dictionaries: Learning Consistent Semantic Structures on 3D Models from Functions}}\\
\textbf{Minhyuk Sung}, Hao Su, Ronald Yu, Leonidas Guibas\\
NeurIPS 2018\\
%\href{https://arxiv.org/abs/1805.09957}{arxiv:1805.09957}\\
\blankline

\item \label{sgp18}
\href{https://mhsung.github.io/fuzzy-set-dual}{\textbf{Learning Fuzzy Set Representations of Partial Shapes on Dual Embedding Spaces}}\\
\textbf{Minhyuk Sung}, Anastasia Dubrovina, Vladimir G. Kim, Leonidas Guibas\\
SGP 2018 (Symposium on Geometry Processing)\\
%Project page: \href{https://mhsung.github.io/fuzzy-set-dual}{https://mhsung.github.io/fuzzy-set-dual}\\
\blankline

\item \label{siggraphasia17}
\href{https://mhsung.github.io/complement-me.html}{\textbf{ComplementMe: Weakly-Supervised Component Suggestions for 3D Modeling}}\\
\textbf{Minhyuk Sung}, Hao Su, Vladimir G. Kim, Siddhartha Chaudhuri, Leonidas Guibas\\
SIGGRAPH Asia 2017\\
{\color{linkcolor}
\textbf{Featured in an ACM SIGGRAPH press release}:
\href{https://www.eurekalert.org/pub_releases/2017-11/afcm-sad120417.php}{[Link 1]}
\href{https://scienmag.com/simplifying-assembly-based-design-for-3-d-modeling/}{[Link 2]}
}\\
%Project page: \href{https://mhsung.github.io/complement-me}{https://mhsung.github.io/complement-me}\\
\blankline

\item \label{siggraphasia15}
\href{https://mhsung.github.io/structure-completion.html}{\textbf{Data-Driven Structural Priors for Shape Completion}}\\
\textbf{Minhyuk Sung}, Vladimir G. Kim, Roland Angst, Leonidas Guibas\\
SIGGRAPH Asia 2015\\
%Project page: \href{https://mhsung.github.io/structure-completion}{https://mhsung.github.io/structure-completion}\\
\blankline

\item \label{icip13}
\href{http://dx.doi.org/10.1109/icip.2013.6738034}{\textbf{Image Unprojection for 3D Surface Reconstruction: A Triangulation-based Approach}}\\
\textbf{Min-Hyuk Sung}, Hwasup Lim, Hyoung-Gon Kim, Sang Chul Ahn\\
IEEE International Conference on Image Processing (ICIP) 2013\\
%doi: \href{http://dx.doi.org/10.1109/icip.2013.6738034}{10.1109/icip.2013.6738034}\\
\blankline

\item \label{cg13}
\href{http://dx.doi.org/10.1016/j.cag.2012.11.002}{\textbf{Finding the M-best Consistent Correspondences between 3D Symmetric Objects}} \\
\textbf{Min-Hyuk Sung} and Junho Kim\\
Computers \& Graphics, Feb.-Apr. 2013.\\
%Computers \& Graphics, Feb.-Apr. 2013 (vol. 37, no. 1-2)\\
%doi: \href{http://dx.doi.org/10.1016/j.cag.2012.11.002}{10.1016/j.cag.2012.11.002}\\
\blankline

\item \label{tvcg12}
\href{http://dx.doi.org/10.1109/TVCG.2012.29}{\textbf{A Triangulation-Invariant Method for Anisotropic Geodesic Map Computation on Surface Meshes}} \\
Sang Wook Yoo, Joon-Kyung Seong, \textbf{Min-Hyuk Sung}, Sung Yong Shin and Elaine Cohen\\
IEEE Transactions on Visualization and Computer Graphics (TVCG), Oct. 2012.\\
%IEEE Transactions on Visualization and Computer Graphics (TVCG), Oct. 2012 (vol. 18, no. 10) \\
%doi: \href{http://dx.doi.org/10.1109/TVCG.2012.29}{10.1109/TVCG.2012.29}\\
\blankline

%\item \label{isuvr12}
%\href{http://dx.doi.org/10.1109/ISUVR.2012.12}{\textbf{Putting Real-World Objects into Virtual World: Fast Automatic Creation of Animatable 3D models with a Consumer Depth Camera}} (The Best Paper Award)\\
%Seong-Oh Lee, Jong-Ho Lee, \textbf{Min-Hyuk Sung}, Young-Woon Cha, Hyoung-Gon Kim, Sang Chul Ahn \\
%International Symposium on Ubiquitous Virtual Reality (ISUVR), 2012\\
%doi: \href{http://dx.doi.org/10.1109/ISUVR.2012.12}{10.1109/ISUVR.2012.12}\\
%\blankline
%
%\item \label{urai11}
%\href{http://dx.doi.org/10.1109/URAI.2011.6145896}{\textbf{Plane-dominant Object Reconstruction for Robotic Spatial Augmented Reality}} \\
%Changwoo Nam, \textbf{Min-Hyuk Sung}, Joo-Haeng Lee, Junho Kim \\
%Ubiquitous Robots and Ambient Intelligence (URAI), 2011 \\
%doi: \href{http://dx.doi.org/10.1109/URAI.2011.6145896}{10.1109/URAI.2011.6145896}\\
%\blankline
%
%\item \label{kjmr11}
%\textbf{Implementing Parallel Tracking and Mapping on Android Phones} \\
%Jinwoo Lee, \textbf{Min-Hyuk Sung}, Kyuman Jeong, Jae-In Hwang, Junho Kim \\
%The 4th Korea-Japan Workshop on Mixed Reality, 2011\\
%\blankline
%
%\item \label{ms10}
%\href{http://library.kaist.ac.kr/thesis02/2010/2010M020084064_S1Ver2.pdf}{
%\textbf{A Spectral Approach to Shape Matching Using a Heat Kernel Function}} \\
%\textbf{Min-Hyuk Sung} \\
%Master's thesis, Korea Advanced Institute of Science and Technology (KAIST), Daejeon, South Korea, 2010 \\
%link: \href{http://library.kaist.ac.kr/thesis02/2010/2010M020084064_S1Ver2.pdf}{http://library.kaist.ac.kr/thesis02/2010/2010M020084064\_S1Ver2.pdf}

\end{enumerate}

\blankline


%----------------------------------------------------------------------------------------
%	HONORS AND SCHOLARSHIPS
%----------------------------------------------------------------------------------------

\section{Honors and Scholarships}

\begin{tabular}{rl}
2019 & Doctoral Consortium\\
& ICCV 2019\\
&\\
2019 & Doctoral Consortium\\
& SIGGRAPH 2019\\
&\\
2013 & Doctoral Study Abroad Scholarship Recipient Honors\\
& Korea Foundation for Advanced Studies (KFAS)\\
&\\
2008-2010 & National Science and Engineering Graduate Research Scholarship\\
& (S2-2008-000-00006-2)\\
& Korea Student Aid Foundation (KOSAF)\\
&\\
2004-2008 & National Science and Engineering Scholarship \\
& Korea Student Aid Foundation (KOSAF)\\
&\\
2005-2008 & Merit-based Scholarship\\
& Korea Advanced Institute of Science and Technology (KAIST)\\
&\\
\end{tabular}\\

\blankline


%----------------------------------------------------------------------------------------
%	TEACHING EXPREENCE
%----------------------------------------------------------------------------------------

\section{Teaching Experience}

\begin{tabular}{r|p{11cm}}
\textsc{2021 Spring} & Instructor\\
& \href{https://mhsung.github.io/courses/kaist-cs492-3dml-2021-spring}{\textbf{CS492(H) Machine Learning for 3D Data}}, KAIST\\
& \\

%------------------------------------------------

\textsc{2018 Spring} & Guest Lecturer\\
& \textbf{CS233 Geometric and Topological Data Analysis}, Stanford\\
& \\

%------------------------------------------------

\textsc{2016 Fall} & Course Assistance\\
& \textbf{CS268 Geometric Algorithms}, Stanford\\
& \\

%------------------------------------------------

\textsc{2015 Fall} & Course Assistance\\
& \textbf{CS348A Computer Graphics: Geometric Modeling}, Stanford\\
& \\

%------------------------------------------------

\textsc{2009 Spring} & Teaching Assistance\\
& \textbf{CS202 Problem Solving}, KAIST\\
\end{tabular}\\

\blankline


%----------------------------------------------------------------------------------------
%	TECHNICAL SKILLS 
%----------------------------------------------------------------------------------------

%\section{Technical Skills}
%
%\begin{tabular}{rl}
%Programming Languages: &C, C++, Java, Objective-C, Python, OpenGL Shader Language\\
%Graphics Libraries: &OpenGL (ver.1.x, 2.x, 3.x), OpenGL ES (ver.1.x, 2.x), OpenCV,\\
%& OpenSceneGraph(OSG), OpenMesh, CGAL\\
%Numerical  Libraries: &BLAS, LAPACK, TAUCS, SuperLU, Eigen, SparseLM, ILOG CPLEX, Matlab\\
%Programming Libraries: &STL, Boost C++, ARM Neon Instructions\\
%GUI Frameworks: &Qt, MFC, Tcl/Tk\\
%Operating Systems: &Windows, Mac OS X, Linux(Ubuntu)\\
%Mobile Platforms: &iOS, Android (including NDK), Unity 3D\\
%\end{tabular}
%
%\blankline

%----------------------------------------------------------------------------------------
%	ACADEMIC ACTIVITIES
%----------------------------------------------------------------------------------------

\section{Academic Activities}

%Reviewer, SIGGRAPH. 2019, 2020.\\
%Reviewer, SIGGRAPH Asia. 2018, 2019, 2020, 2021.\\
%Reviewer, NeurIPS. 2020.\\
%Reviewer, ICML. 2021.\\
%Reviewer, ICLR. 2021, 2022.\\
%Reviewer, CVPR. 2021.\\
%Reviewer, ICCV. 2021.\\
%Reviewer, ACCV. 2020.\\
%Reviewer, WACV. 2021, 2022.\\
%Reviewer, 3DV. 2017, 2018, 2019, 2020.\\
%Reviewer, IEEE Robotics and Automation Letters. 2019.\\
%Reviewer, IEEE Transactions on Pattern Analysis and Machine Intelligence. 2019, 2020.\\
%Reviewer, IEEE Transactions on Visualization and Computer Graphics. 2019, 2020.\\
%Reviewer, Eurographics. 2021.\\
%Reviewer, Pacific Graphics. 2016, 2018, 2019, 2020.\\
%Reviewer, ACM Transactions on Graphics, 2020.\\
%Reviewer, Computer Graphics Forum. 2018, 2019, 2020.\\
%Reviewer, The Visual Computer. 2009, 2016, 2019.\\
%Reviewer, Computers \& Graphics. 2012, 2019.\\
%Reviewer, IEEE Virtual Reality. 2012.\\
%Reviewer, Computer Animation and Virtual Worlds. 2011.\\

\begin{tabular}{rl}
Program Committee & Eurographics 2022 \\
Reviewer & SIGGRAPH, SIGGRAPH Asia, Eurographics, Pacific Graphics, \\
  & CVPR, ICCV, 3DV, WACV, ACCV, \\
  & NeurIPS, ICML, ICLR, \\
  & ACM TOG, IEEE TVCG, CGF, TVC, C\&G, IEEE TPAMI, IEEE RA-L. \\
Co-Organizer & \href{https://geometry.stanford.edu/struco3d/}{Structural and Compositional Learning on 3D Data}, Workshop at ICCV 2021. \\
\end{tabular}\\

\blankline


%----------------------------------------------------------------------------------------
%	ACADEMIC ACTIVITIES
%----------------------------------------------------------------------------------------

\section{Talks}
\begin{tabular}{rl}
Oct 2021 & Samsung Electronics. Manufacturing Technology Center. Seminar Speaker.\\
Oct 2021 & Korea University. School of Biomedical Engineering. Colloquium Speaker.\\
Sep 2021 & GIST. School of Integrated Technology. Seminar Speaker.\\
Jul 2021 & Korea Computer Graphics Society 2021. Invited Speaker.\\
May 2021 & KAIST. Graduate School of Computing. Colloquium Speaker.\\
Apr 2021 & KAIST. Software Graduate Program. Colloquium Speaker.\\
Feb 2021 & Korean Computer Vision Society. Computer Vision Researcher Forum Speaker.\\
Feb 2021 & KAIST. Graduate School of Culture Technology. Invited Speaker.\\
Jan 2021 & Kakao Brain. Invited Speaker.\\
\end{tabular}\\

\blankline


%----------------------------------------------------------------------------------------
%	REFERENCES
%----------------------------------------------------------------------------------------

\section{References}

\renewcommand{\arraystretch}{1.5}
\begin{tabular}{lll}

\href{https://geometry.stanford.edu/member/guibas/}{\textbf{Leonidas Guibas}} &
Professor, Stanford University &
guibas@cs.stanford.edu\\

\href{http://www.vovakim.com/}{\textbf{Vladimir Kim}} &
Senior Research Scientist, Adobe Research &
vokim@adobe.com\\

%\multirow{2}{*}{\href{https://www.cse.iitb.ac.in/~sidch/}{\textbf{Siddhartha Chaudhuri}}} &
\href{https://www.cse.iitb.ac.in/~sidch/}{\textbf{Siddhartha Chaudhuri}} &
\makecell[l]{Assistant Professor, IIT Bombay \\ Senior Research Scientist, Adobe Research} &
\makecell[l]{sidch@cse.iitb.ac.in \\ sidch@adobe.com} \\

\href{http://www.cs.sfu.ca/~haoz/}{\textbf{Hao (Richard) Zhang}} &
Professor, Simon Fraser University &
haoz@cs.sfu.ca\\

\end{tabular}

\end{document}
